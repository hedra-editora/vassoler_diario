\begin{itemize}


\item \textbf{Diário de um escritor na Rússia} é um livro composto por textos que caminham na
fronteira entre a crônica, a ficção e o ensaio. Para escrevê"-lo, Flávio Ricardo Vassoler
viajou por 10 cidades da Rússia europeia que receberam jogos da Copa do Mundo de
2018: Moscou, Níjni Novgorod, Kazan, Saransk, Samara, Volgogrado (antiga
Stalingrado), Rostov"-sobre"-o"-Don, Sotchi, Kaliningrado (antiga Königsberg) e São
Petersburgo. Em diálogo com a literatura e os costumes, a cultura e a história do país
cujos caminhos e descaminhos moldaram o transcurso do século \versal{XX}, a obra funde
estórias e histórias para compor um panorama atualíssimo da Rússia que desemboca na
protodinastia de Vladímir Putin.
  
\item \textbf{Flávio Ricardo Vassoler} é doutor em Teoria Literária e Literatura Comparada pela \versal{FFLCH-USP}, com pós"-doutorado em Literatura Russa pela Northwestern University (\versal{EUA}). É autor das obras \emph{O evangelho segundo Talião} (nVersos, 2013), \emph{Tiro de misericórdia} (nVersos, 2014) e \emph{Dostoiévski e a dialética: Fetichismo da forma, utopia como conteúdo} (Hedra, 2018), além de ter organizado o livro de ensaios \emph{Fiódor Dostoiévski e Ingmar Bergman: O niilismo da modernidade} (Intermeios, 2012) e, ao lado de Alexandre Rosa e Ieda Lebensztayn, o livro \emph{Pai contra mãe e outros contos} (Hedra, 2018), de Machado de Assis. Escreve para o caderno literário ``Aliás'', do jornal \emph{O Estado de S. Paulo}, para o caderno ``Ilustríssima'', do jornal \emph{Folha de S.Paulo}, e para as revistas \emph{Veja} e \emph{Carta Capital}. 

\end{itemize}

