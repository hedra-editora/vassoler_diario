\begin{itemize}


\item \textbf{Diário de um escritor na Rússia} \lipsum[1]
  
\item \textbf{Flávio Ricardo Vassoler} é doutor em Teoria Literária e Literatura Comparada pela \versal{FFLCH-USP}, com pós"-doutorado em Literatura Russa pela Northwestern University (\versal{EUA}). É autor das obras \emph{O evangelho segundo Talião} (nVersos, 2013), \emph{Tiro de misericórdia} (nVersos, 2014) e \emph{Dostoiévski e a dialética: Fetichismo da forma, utopia como conteúdo} (Hedra, 2018), além de ter organizado o livro de ensaios \emph{Fiódor Dostoiévski e Ingmar Bergman: O niilismo da modernidade} (Intermeios, 2012) e, ao lado de Alexandre Rosa e Ieda Lebensztayn, o livro \emph{Pai contra mãe e outros contos} (Hedra, 2018), de Machado de Assis. Escreve para o caderno literário ``Aliás'', do jornal \emph{O Estado de S. Paulo}, para o caderno ``Ilustríssima'', do jornal \emph{Folha de S.Paulo}, e para as revistas \emph{Veja} e \emph{Carta Capital}. 

\end{itemize}

