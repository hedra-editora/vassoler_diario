\chapter*{}

\vspace*{\fill}

\thispagestyle{empty}

\epigraph{Essa aparente desordem que, em essência, é a ordem burguesa no mais alto grau.}{Fiódor Dostoiévski, \emph{Notas de inverno sobre impressões~de~verão}\footnotemark}

\footnotetext{Fiódor Dostoiévski, \emph{Notas de inverno sobre impressões de verão}. Tradução de Boris Schnaiderman. São Paulo: Editora 34, 2000, p. 113.}

\epigraph{Come ananás, mastiga perdiz\\ Teu dia está prestes, burguês.}{Vladímir Maiakóvski, ``Come ananás''\footnotemark}

\footnotetext{Vladímir Maiakóvski, ``Come ananás''. In: \emph{Poemas}. Tradução de Augusto e Haroldo de Campos e Boris Schnaiderman. São Paulo: Perspectiva, 2017, p. 135.}

\epigraph{Quem não lamenta o fim da União Soviética não tem coração, mas aquele que quer restaurá"-la não tem cérebro.}{Assim falou o neotsar russo Vladímir Putin\footnotemark}

\footnotetext{Deparei com a citação de Putin como epígrafe da obra \emph{Limonov}, do escritor francês Emmanuel Carrère. Tradução de Sérgio Teles. Rio de Janeiro: Alfaguara, 2016, p. 7.}